\documentclass[12pt]{article}
\usepackage[utf8]{inputenc}
\usepackage[
    top=        2.5cm, 
    bottom=     2.5cm, 
    left=       2.5cm, 
    right=      2.5cm
]{geometry}
% \usepackage{titling}
\usepackage{amsmath}
\usepackage{amssymb}
\usepackage{enumitem}

\input{../../problem.tex}
% --- solutionbox.tex ---
\usepackage[many]{tcolorbox}

\tcbuselibrary{skins, breakable}

% Define the "solution" environment
\newtcolorbox{solution}{
  enhanced,
  colback=black!3,
  colframe=black!50,
  boxrule=0.5pt,
  arc=2mm,
  left=5mm,
  right=5mm,
  top=2mm,
  bottom=2mm,
  fonttitle=\bfseries\sffamily,
  title=Solution,
  colbacktitle=black!10,
  coltitle=black,
  breakable,
  before skip=2ex,
  after skip=2ex,
  before upper=\strut, 
  detach title
}


\begin{document}
\noindent
AMS 301\\
Ma, Pei \\
Set 7\\

Sec 7.1: 4, 6ab, 7, 11, 12, 15, 19, 28, 30\\
Sec 7.3: 1 ,2, 3a\\

Section 7.1\\
4. 
\begin{enumerate}[label=(\alph*)]
\item Find a recurrence relation for the number of ways to go n miles by foot
walking at 2 miles per hour or jogging at 4 miles per hour or running at 8 miles
per hour; at the end of each hour a choice is made of how to go the next hour.
\begin{solution}
    $a_{n-2}+ a_{n-4} + a_{n-8} = a_n$\\
    $a_0 = 1, a_i = 0$ for $i\le0$
\end{solution}
\item How many ways are there to go 12 miles. 
\begin{solution}
    $a_2 = a_0 + a_{-2} + a_{-6} = 1$\\
    $a_4 = a_0 + a_2 + a_{-4} = 2$\\
    $a_6 = a_4 + a_2 + a_{-2} = 2 + 1 = 3$\\
    $a_8 = a_6 + a_4 + a_0 = 3 + 2 = 5$\\
    $a_{10} = a_8 + a_6 + a_2 = 5 + 3 + 1 = 9$\\
    $a_{12} = a_{10} + a_8 + a_4 = 9 + 5 + 2 = 16$
\end{solution}
\end{enumerate}

6. 
\begin{enumerate}[label=(\alph*)]
\item Find a recurrence relation for the number of n-digit binary 
sequences with no pair of consecutive 1s.
\begin{solution}
    $a_0 = 1, (\emptyset); a_1 = 2, (0, 1); a_2 = 3, (00, 01, 10); 
    a_3 = 5,(000, 001, 010, 100, 101)$\\
    $a_n = a_{n-1} + (n-1)$; \\
    ($a_{n-1}$) represent adding a 0 to the end of 
    each term in $a_{n-1}$ and $(n-1)$ represent $n-1$ 
    possible terms in $n-1$ digit binary sequence that ends with 1. 
\end{solution}
\item Repeat for n-digit ternary sequences.
\begin{solution}
    $a_0 = 1, (\emptyset); a_1 = 3, (0, 1, 2); a_2 = 8,
    (00, 01, 02, 10, 12, 20, 21, 22)$ \\
    $a_n = 2\times a_{n-1} + (n-1)$ \\
    Follows similar logic to the previous question, but since we have the 
    option of adding a 0 or 2 to the end of all $n-1$ digit outcomes, we
    have $2\times a_{n-1}$. 
\end{solution}
\end{enumerate}
7. Find a recurrence relation for the number of pairs of rabbits after 
$n$ months if (1) initially there is one pair of rabbits who were just 
born, and (2) every month each pair of rabbits that are over one month 
old have a pair of offspring (a male and a female).
\begin{solution}
    $a_0 = 1$ pair; $a_1 = 1$ pair\\
    $a_n = a_{n-1} + a_{n-2}$
\end{solution}
11. Find a recurrence relation for an for the number of ways for an 
image to be reflected $n$ times by internal faces of two adjacent panes 
of glass. The diagram below shows that $a_0 = 1$, $a_1 = 2$, and 
$a_2 = 3$.
\begin{solution}
    In the internal faces of two adjacent panes, there are 3 lines 
    which the image can be reflected on. We can label them 0, 1, 2. 
    It is possible to reflect under 0, above and under 1, and above 2. 
    We assume the image to begin from above, making the only possible 
    reflections to go from above. \\
    For 0 reflections, there is only $a_0 = 1$ way for the image to be 
    reflected, which is no ways. \\
    For 1 reflection, the image can reflect
    against line 1 and 2 from above, which is $a_1 = 1[1,above] + 
    1[2,above] = 2$ ways. \\
    For 2 reflections, we include the previous count, 2, which hits 
    line 0 from below from line 2, and add 1 for the reflections which 
    hits line 0 from below from line 1. \\
    So, following similar logic, for $n$ reflections, there are 
    $a_n = a_{n-1} + a_{n-2}$ reflections. 
\end{solution}
12. Find a recurrence relation for the number of regions created by $n$ 
mutually intersecting circles on a piece of paper (no three circles 
have a common intersection point).
\begin{solution}
    Let $a_n$ be the number of regions created by $n$ mutually 
    interecting circles. $a_0 = 1$ region; $a_1 = 2$ region; $a_2 = 4$ 
    intersections; $a_3 = 8$ intersections. \\
    From a little bit of intuition, adding another intersection will 
    create $2n$ the amount of regions and add an additional region. So, 
    $a_n = a_{n-1} + 2(n-1)$ regions. 
\end{solution}
15. Find a recurrence relation for the amount of money in a savings 
account after $n$ years if the interest rate is 6 percent and \$50 
is added at the start of each year.
\begin{solution}
    $a_n = 1.06(a_{n-1} + 50)$ with initial balance  $a_0$. 
\end{solution}
19. 
\begin{enumerate}[label=(\alph*)]
\item Find a recurrence relation for the number of sequences of 
1s, 3s, and 5s whose terms sum to n.
\begin{solution}
    $a_n = a_{n-1} + a_{n-3} + a_{n-5}$, with initial condition 
    $a_0 = 1$. 
\end{solution}
\item Repeat part (a) with the added condition that no 5 can be 
followed by a 1.
\begin{solution}
    Let $a_n = x_n + y_n$ where $x_n$ is the number of valid 
    sequences where the last term is not 5 and $y_n$ is the 
    number of valid sequences where the last term is 5. \\
    For $x_n$, if the previous term ended with 1, then we have 
    then the previous sequence must be $x_{n-1}$. \\
    If the previous term ended with 3, then the previous sequence
    is $x_{n-3} + y_{n-3}$ \\
    So, we can write $x_n = x_{n-1} + (x_{n-3} + y_{n-3})$. \\
    For $y_n$, it can follow any valid sequence summing to $n-5$, 
    $y_n = x_{n-5} + y_{n-5}$.\\\\
    If we combine both terms, \\
    $a_n = x_{n-1} + (x_{n-3} + y_{n-3}) 
    + (x_{n-5} + y_{n-5}) = x_{n-1} + a_{n-3} + a_{n-5}$.\\
    $y_{n-1} = a_{n-6}$ \\
    $x_{n-1} = a_{n-1} - y_{n-1} \rightarrow x_{n-1}
    = a_{n-1} - a_{n-6}$. \\\\
    So, the final recurrence relation is $a_n = a_{n-1}+a_{n-3}
    +a_{n-5}-a_{n-6}$. 
\end{solution}
\item Repeat part (a) with the condition of no subsequence of 135.
\begin{solution}
    The subsequence 135 has a sum of 9. So, for any sequnce 
    $0\le n<9$, no sequence is excluded. We can follow the same 
    formula as in part (a) $a_n = a_{n-1}+a_{n-3}+a_{n-5}$. Then, 
    we must remove the only possible sequence with a sum of 9 that 
    includes the subsequence $135$, which is itself. This is 
    done by $a_n = a_{n-1} + a_{n-3} + a_{n-5} - a_{n-9}$.
\end{solution}
\end{enumerate}
28. Find a recurrence relation for the number of ways to pick $k$ 
objects with repetition from $n$ types.
\begin{solution}
    $a_{n, 0} = 1$ for $n\ge0$, there is 1 way to pick 0 items from 
    $n$ types.\\
    $a_{0, k} = 0$, there are no ways to pick $k$ item from 0 types. \\
    $a_{n, k} = a_{n-1,k} + a_{n, k-1}$\\
    $a_{n-1, k}$ is how many ways to pick if there was one less type. 
    If we add another type, we can just pick that 0 times. Now, we need 
    to consider $a_{n, k-1}$ which is where we pick at least 1 object 
    of type $n$ which is the new type. 
\end{solution}
30. Find a recurrence relation for $a_{n,k}$ , the number of ways 
to order $n$ doughnuts from $k$ different types of doughnuts if two 
or four or six doughnuts must be chosen of each type.
\begin{solution}
    $a_{0,0} = 1$\\ 
    $a_{n,0} = 0$ for $n>0$ since you cannot have donuts 
    if you are picking from none.\\
    $a_{n, k} = 0$ for $n<2k$ since each type must contribute
    two doughnuts. \\
    $a_{n,k} = a_{n-2,k-1} + a_{n-4,k-1} + a_{n-6,k-1}$
\end{solution}
Section 7.3\\
1. If \$500 is invested in a savings account earning $8$ percent a 
year, give a formula for the amount of money in the account after 
$n$ years.
\begin{solution}
    $a_n = 500\cdot1.08^n$
\end{solution}
2. Find and solve a recurrence relation for the number of $n$-digit 
ternary sequences with no consecutive digits being equal.
\begin{solution}
    $a_n = 2a_{n-1}$ and $a_1=3$. $a_n = 3\cdot 2^{n-1}$
\end{solution}
3. Solve the following recurrence relations:
\begin{enumerate}[label=(\alph*)]
\item $a_n = 3a_{n-1} + 4a_{n-2}, a_0 = a_1 = 1$
\begin{solution}
    Get the characteristic equation: $r^2-3r-4=0$, which we can solve
    and get $r=4, -1$. \\
    $a_n= \alpha_1 \cdot 4^n + \alpha_2 \cdot (-1)^n$. \\
    $a_0 = \alpha_1 + \alpha_2 = 1$ and 
    $a_1 = \alpha_1 \cdot 4^1 + \alpha_2 \cdot (-1)^1 = 1$\\
    We get $5\alpha_1 = 2 \rightarrow \alpha_1 = \frac{2}{5}$ and
    $\alpha_2 = \frac{3}{5}$. \\
    So, the recurrenc relation is equal to $a_n=\frac{2}{5}\cdot4^n + 
    \frac{3}{5}\cdot(-1)^n$. 
\end{solution}
\end{enumerate}
\end{document}