\documentclass[12pt]{article}
\usepackage[utf8]{inputenc}
\usepackage[
    top=        2.5cm, 
    bottom=     2.5cm, 
    left=       2.5cm, 
    right=      2.5cm
]{geometry}
% \usepackage{titling}
\usepackage{amsmath}
\usepackage{amssymb}
\usepackage{enumitem}

\usepackage{titlesec}
\usepackage[many]{tcolorbox}

% Adjust spacing after the chapter title
\titlespacing*{\chapter}{0cm}{-2.0cm}{0.50cm}
\titlespacing*{\section}{0cm}{0.50cm}{0.25cm}

% Indent 
\setlength{\parindent}{0pt}
\setlength{\parskip}{1ex}

% --- Theorems, lemma, corollary, postulate, definition ---
% \numberwithin{equation}{section}

\newtcbtheorem[]{problem}{Solution}%
{enhanced,
	colback=black!5,
	colbacktitle=black!10,
	coltitle=black,
	boxrule=0pt,
	frame hidden,
	borderline west={0.5mm}{0.0mm}{black},
	fonttitle=\bfseries\sffamily,
	breakable,
	before skip=3ex,
	after skip=3ex,
	detach title
}{problem}

\tcbuselibrary{skins, breakable}

% --- You can define your own color box. Just copy the previous \newtcbtheorm definition and use the colors of yout liking and the title you want to use.
% --- solutionbox.tex ---
\usepackage[many]{tcolorbox}

\tcbuselibrary{skins, breakable}

% Define the "solution" environment
\newtcolorbox{solution}{
  enhanced,
  colback=black!3,
  colframe=black!50,
  boxrule=0.5pt,
  arc=2mm,
  left=5mm,
  right=5mm,
  top=2mm,
  bottom=2mm,
  fonttitle=\bfseries\sffamily,
  title=Solution,
  colbacktitle=black!10,
  coltitle=black,
  breakable,
  before skip=2ex,
  after skip=2ex,
  before upper=\strut, 
  detach title
}


\begin{document}
\noindent
AMS 301\\
Ma, Pei \\
Set 7\\

Sec 5.4: 16, 19, 21, 22, 36, 38, 42c\\
Sec 6.1: 2b, 4bc, 6, 8, 10, 16\\
Sec 6.2: 2, 20, 22\\

Section 5.4\\ 
16. Consider the problem of distributing 10 distinct books among three 
different people with each person getting at least one book. Explain why 
the following solution strategy is wrong: first select a book to give to 
the first person in 10 ways; then select a book to give to the second 
person in nine ways; then select a book to give to the third person in 
eight ways; and finally distribute the remaining seven books in $7^3$ ways.
\begin{solution}
    There are two issues with this method. \\
    First, to distribute the remaining 7 books, there would be $3^7$ ways, 
    as this is a cases of distributing 7 distinct objects to 3 different 
    people. \\
    Second, it does not account 
    for repeating cases where the first book each person gets is repeated 
    when distributing the remaining 7 books. For example, lets say person
    1 gets book 1. Then, person 2 and 3 gets books 2 and 3 respectively. 
    The final distribution is that person 1 gets 1, 4, 5, and 6. It does 
    not matter what persons 2 or 3 gets. This 
    distribution gets repeated if person 1 gets book 4 as their first book, 
    then gets books 1, 5, and 6 when distribution the rest of the 7 books, 
    and we assume persons 2 and 3 gets the same books to illustrate the 
    case of repeating cases. 
\end{solution}
19. How many ways are there to distribute three different teddy bears and 
nine identical lollipops to four children
\begin{enumerate}[label=(\alph*)]
    \item Without restriction?
    \begin{solution}
        Distribute 3 distinct objects to 4 children, so there are 
        $4^3 = 64$ distributions. Then, to distribute 9 identical objects 
        to 4 different children, there are 
        $C(9+4-1, 9)=\frac{12!}{3!9!}=220$ 
        distributions. So, there are a total of $64\times220 = 14080$ 
        distributions.
    \end{solution}
    \item With no child getting two or more teddy bears?
    \begin{solution}
        We can only give a child at most 1 teddy bear, which implies 
        that there will always be one child, so there will be 
        $4*(3*2*1)=24$ distributions. Then, there are a total of 
        $24\times220=5280$ distributions.  
    \end{solution}
    \item With each child getting three “goodies”?
    \begin{solution}
        We only need to distribute the distinct objects. So, there 
        will be $4^3 = 64$ distributions. The way the lollipops are 
        distributed is trivial since they're identical. 
    \end{solution}
\end{enumerate}
21. Suppose a coin is tossed 12 times and there are three heads and nine 
tails. How many such sequences are there in which there are at least 
five tails in a row?
\begin{solution}
    Let can describe the scenarios with x being tails and $\vert$ being
    heads. So, we can write xx$\vert$xxxxx$\vert$x$\vert$x. 
    $x_1+x_2+x_3+x_4=9$ where one $x_i\ge5$ and that $x_i = z_i + 5$. 
    $C(9-(5)+4-1, 4-1) = \frac{(9-5+4-1)!}{(7-4+1)!(4-1)!} = 35$. There 
    are 4 sections that can have the sequence 
    of 5 ore more tails, so the total distributions is $35\times4=140$. 
\end{solution}
22. How many binary sequences of length 20 are there that
\begin{enumerate}[label=(\alph*)]
    \item Start with a run of 0s—that is, a consecutive sequence of 
    (at least) one 0— then a run of 1s, then a run of 0s, then a run 
    of 1s, and finally finish with a run of 0s?
    \begin{solution}
        Each sequence will look like 0\_1\_0\_1\_0\_. We have to distribute
        the remaining 15 digits into 5 boxes. $C(15+5-1, 5-1)=3876$ 
        distributions. 
    \end{solution}
    \item Repeat part (a) with the constraint that each run is of length 
    at least 2.
    \begin{solution}
        Each sequence will look like 00\_11\_00\_11\_00\_. We have to 
        distribute the remaining 10 digits into 5 boxes. 
        $C(10+5-1, 5-1) = 1001$ distributions. 
    \end{solution}
\end{enumerate}
36. How many election outcomes in the race for class president are there 
if there are five candidates and 40 students in the class and
\begin{enumerate}[label=(\alph*)]
    \item Every candidate receives at least two votes?
    \begin{solution}
        
    \end{solution}
    \item One candidate receives at most one vote and all the others receive at least
    two votes?
    \item No candidate receives a majority of the votes?
    \item Exactly three candidates tie for the most votes?
\end{enumerate}
38. How many integer solutions are there to the equation $x_1 + x_2 + x_3 + x_4 \le 15$
with $x_i \ge -10$?
\begin{solution}
    
\end{solution}
42. How many nonnegative integer solutions are there to 
$x_1 + x_2 + \dots + x_5 = 20$\\
(c) With $x_1=2x_2$
\begin{solution}
    
\end{solution}
Section 6.1\\
2. Build a generating function for ar, the number of integer solutions 
to the following equations:
(b) $e_1 + e_2 + e_3 = r, 0 < e_i < 6$
\begin{solution}
    
\end{solution}
4. Build a generating function for ar, the number of distributions of r identical
objects into
(a) Five different boxes with at most three objects in each box
(b) Three different boxes with between three and six objects in each box
(c) Six different boxes with at least one object in each box
(d) Three different boxes with at most five objects in the first box
Section 6.2
\end{document}