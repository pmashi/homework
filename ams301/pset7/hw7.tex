\documentclass[12pt]{article}
\usepackage[utf8]{inputenc}
\usepackage[
    top=        2.5cm, 
    bottom=     2.5cm, 
    left=       2.5cm, 
    right=      2.5cm
]{geometry}
% \usepackage{titling}
\usepackage{amsmath}
\usepackage{amssymb}
\usepackage{enumitem}

\input{../../problem.tex}
% --- solutionbox.tex ---
\usepackage[many]{tcolorbox}

\tcbuselibrary{skins, breakable}

% Define the "solution" environment
\newtcolorbox{solution}{
  enhanced,
  colback=black!3,
  colframe=black!50,
  boxrule=0.5pt,
  arc=2mm,
  left=5mm,
  right=5mm,
  top=2mm,
  bottom=2mm,
  fonttitle=\bfseries\sffamily,
  title=Solution,
  colbacktitle=black!10,
  coltitle=black,
  breakable,
  before skip=2ex,
  after skip=2ex,
  before upper=\strut, 
  detach title
}


\begin{document}
\noindent
AMS 301\\
Ma, Pei \\
Set 7\\

Sec 5.4: 16, 19, 21, 22, 36, 38, 42c\\
Sec 6.1: 2b, 4bc, 6, 8, 10, 16\\
Sec 6.2: 2, 20, 22\\

Section 5.4\\ 
16. Consider the problem of distributing 10 distinct books among three 
different people with each person getting at least one book. Explain why 
the following solution strategy is wrong: first select a book to give to 
the first person in 10 ways; then select a book to give to the second 
person in nine ways; then select a book to give to the third person in 
eight ways; and finally distribute the remaining seven books in $7^3$ ways.
\begin{solution}
    There are two issues with this method. \\
    First, to distribute the remaining 7 books, there would be $3^7$ ways, 
    as this is a cases of distributing 7 distinct objects to 3 different 
    people. \\
    Second, it does not account 
    for repeating cases where the first book each person gets is repeated 
    when distributing the remaining 7 books. For example, lets say person
    1 gets book 1. Then, person 2 and 3 gets books 2 and 3 respectively. 
    The final distribution is that person 1 gets 1, 4, 5, and 6. It does 
    not matter what persons 2 or 3 gets. This 
    distribution gets repeated if person 1 gets book 4 as their first book, 
    then gets books 1, 5, and 6 when distribution the rest of the 7 books, 
    and we assume persons 2 and 3 gets the same books to illustrate the 
    case of repeating cases. 
\end{solution}
19. How many ways are there to distribute three different teddy bears and 
nine identical lollipops to four children
\begin{enumerate}[label=(\alph*)]
    \item Without restriction?
    \begin{solution}
        Distribute 3 distinct objects to 4 children, so there are 
        $4^3 = 64$ distributions. Then, to distribute 9 identical objects 
        to 4 different children, there are 
        $C(9+4-1, 9)=\frac{12!}{3!9!}=220$ 
        distributions. So, there are a total of $64\times220 = 14080$ 
        distributions.
    \end{solution}
    \item With no child getting two or more teddy bears?
    \begin{solution}
        We can only give a child at most 1 teddy bear, which implies 
        that there will always be one child, so there will be 
        $4*(3*2*1)=24$ distributions. Then, there are a total of 
        $24\times220=5280$ distributions.  
    \end{solution}
    \item With each child getting three “goodies”?
    \begin{solution}
        We only need to distribute the distinct objects. So, there 
        will be $4^3 = 64$ distributions. The way the lollipops are 
        distributed is trivial since they're identical. 
    \end{solution}
\end{enumerate}
21. Suppose a coin is tossed 12 times and there are three heads and nine 
tails. How many such sequences are there in which there are at least 
five tails in a row?
\begin{solution}
    Let can describe the scenarios with x being tails and $\vert$ being
    heads. So, we can write xx$\vert$xxxxx$\vert$x$\vert$x. 
    $x_1+x_2+x_3+x_4=9$ where one $x_i\ge5$ and that $x_i = z_i + 5$. 
    $C(9-(5)+4-1, 4-1) = \frac{(9-5+4-1)!}{(7-4+1)!(4-1)!} = 35$. There 
    are 4 sections that can have the sequence 
    of 5 or more tails, so the total sequences is $35\times4=140$. 
\end{solution}
22. How many binary sequences of length 20 are there that
\begin{enumerate}[label=(\alph*)]
    \item Start with a run of 0s—that is, a consecutive sequence of 
    (at least) one 0— then a run of 1s, then a run of 0s, then a run 
    of 1s, and finally finish with a run of 0s?
    \begin{solution}
        Each sequence will look like 0\_1\_0\_1\_0\_. We have to 
        distribute the remaining 15 digits into 5 boxes. 
        $C(15+5-1, 5-1)=3876$ binary sequences. 
    \end{solution}
    \item Repeat part (a) with the constraint that each run is of length 
    at least 2.
    \begin{solution}
        Each sequence will look like 00\_11\_00\_11\_00\_. We have to 
        distribute the remaining 10 digits into 5 boxes. 
        $C(10+5-1, 5-1) = 1001$ binary sequences. 
    \end{solution}
\end{enumerate}
36. How many election outcomes in the race for class president are there 
if there are five candidates and 40 students in the class and
\begin{enumerate}[label=(\alph*)]
    \item Every candidate receives at least two votes?
    \begin{solution}
        Distribute two votes to every candidate, and now there are 
        30 remaining undistributed votes. $C(30+5-1, 5-1)=46376$ 
        election outcomes. 
    \end{solution}
    \item One candidate receives at most one vote and all the others 
    receive at least two votes?
    \begin{solution}
        There are 5 possibilities of one candidate receiving at most one 
        vote and the others receiving two. 
        $5\times C(31+5-1, 5-1)=5\times52360=261800$
    \end{solution}
    \item No candidate receives a majority of the votes?
    \begin{solution}
        For a candidate to get major vote, they must get more than 20 
        votes. So, the distributions that a candidate gets majority 
        of the votes is $5\times C(40-21 + 5 - 1, 40-21)=44275$ 
        distributions. There are a total of $C(40 + 5-1, 40)=135751$ 
        distributions. So, the total of distributions where no 
        candidate receives a majority of the votes is $135751-8855=91476$.
    \end{solution}
    \item Exactly three candidates tie for the most votes?
    \begin{solution}
        The way we distribution of the three candidates that get the most
        votes is $C(5, 3) = 10$. \\
        $x_1 + x_1 + x_1 + x_2 x_3 = 3x_1 + x_2 +x_3 = 40$, 
        where $x_1 > x_2 \ge x_3 \ge 0$. If we set $3x_1 \le 40$, we get 
        that $x_1 \le 13$. We can set $x_2 = x_3$ to find
        the minimum $x_1$. $2x_2 = 40-3x_1$ If everyone tied, then each 
        candidate would have 8 votes; this means the minimum $x_1$ must 
        be is 9. Now, we can split each distribution of $x_1$ votes into 
        cases. \\
        Case 1: $x_1 = 9$ \\
        $x_2 + x_3 = 40-3\times9 = 13$, but $x_2, x_3 < 9$. 
        Consider $(x_2, x_3) = (8, 5)$, then there are 4 
        distributions, where $x_2, x_3 \in (5, 8)$. \\
        Case 2: $x_1 = 10$\\ 
        $x_2 + x_3 = 10$ and $x_2, x_3 < 10$. There will be 9 
        distributions. \\
        Case 3: $x_1 = 11$\\
        $x_2 + x_3 = 7$, and $x_2, x_3 < 11$ There will be $C(7+2-1, 7)=8$ 
        distributions.\\ 
        Case 4: $x_1 = 12$\\
        $x_2 + x_3 = 4$, there are 5 distributions.\\
        Case 5: $x_1 = 13$\\
        $x_2 + x_3 = 1$, there are 2 distributions.\\
        So, the total number of distributions where exactly three 
        candidates tie for most votes is $10(4 + 9 + 8 + 5 + 2)=280$.
    \end{solution}
\end{enumerate}
38. How many integer solutions are there to the equation $x_1 + x_2 + x_3 
+ x_4 \le 15$
with $x_i \ge -10$?
\begin{solution}
    $z_i = x_i + 10 \rightarrow x_i = z_i - 10$. $z_i - 10 \ge -10 
    \rightarrow z_i \ge0$.
    We get the equation, 
    $z_1 + z_2 + z_3 + z_4 - 40 \le 15$ or 
    $z_1 +\dots + z_4 =\le 55$. \\
    We can change this to $z_1 + \dots + z_5 = 55$, where $z_5\ge0$. 
    So, there will be $C(55+5-1, 55) = 455126$ integer solutions. 
\end{solution}
42. How many nonnegative integer solutions are there to 
$x_1 + x_2 + \dots + x_5 = 20$\\
(c) With $x_1=2x_2$
\begin{solution}
    $3x_2 + x_3 + x_4 + x_5 = 20 \rightarrow x_3 + x_4 + x_5 = 20-3x_2$\\
    So, $x_2 \in(0, 6)$. \\
    The total number of solutions is 
    $\sum_{x_2=0}^{6} C((20-3x_2) + 3 - 1, 3-1) = 672$ total solutions. 
\end{solution}
Section 6.1\\
2. Build a generating function for $a_r$, the number of integer solutions 
to the following equations:
(b) $e_1 + e_2 + e_3 = r, 0 < e_i < 6$
\begin{solution}
    $1 \le e_i \le 5 \rightarrow g(x)=(x+x^2+x^3+x^4+x^5)^3$
\end{solution}
4. Build a generating function for $a_r$, the number of distributions of 
$r$ identical objects into
\begin{enumerate}[label=(\alph*)]
    \setcounter{enumi}{1}
    \item Three different boxes with between three and six objects in 
    each box
    \begin{solution}
        $e_1 + e_2 + e_3 = r, 3 \le e_i \le 6$\\
        $g(x) = (x^3 + x^4 + x^5 + x^6)^3$
    \end{solution}
    \item Six different boxes with at least one object in each box
    \begin{solution}
        $e_1 + \dots + e_6 = r, e_i \ge 1$\\
        $g(x) = (x + x^2 + \dots )^6 = \dfrac{x^6}{(1-x)^6}$
    \end{solution}
\end{enumerate}
6. Use a generating function for modeling the number of different 
selections of $r$ hot dogs when there are four types of hot dogs.
\begin{solution}
    Let $e_1, e_2, e_3, e_4$ be the number of each type of hot dog 
    selected. Then, we can write the equation $e_1 + e_2 + e_3 + e_4 
    = r$, $e_i \ge0$.\\
    $g(x) = (1+x+x^2+\dots)^4$
    The number of ways to select $r$ hotdogs from 4 types is $C(r+4-1,r)$.
\end{solution}
8. 
\begin{enumerate}[label=(\alph*)]
    \item Use a generating function for modeling the number of 
    different election outcomes in an election for class president 
    if 25 students are voting among four candidates. 
    Which coefficient do we want?
    \begin{solution}
        In a general case, there are a total of $r$ votes in the 
        election amongst 4 candidates. So, we can write, 
        $e_1+e_2+e_3+e_4 = r$, where $e_i \ge0$. Then, the generating
        function is $g(x)=(1+x+x^2+\dots)^4$ \\
        We want to coefficient for $x^{25}$. \\
        $a_{25}=C(25+4-1, 4-1)=3276$.  
    \end{solution}
    \item Suppose each student who is a candidate votes for herself 
    or himself. Now what is the generating function and the required 
    coefficient?
    \begin{solution}
        $e_1+e_2+e_3+e_4=r$, where $e_i\ge 1$. \\
        $g(x) = (x+x^2+\dots)^4$\\
        We want the coefficient for $x^{21}$. $a_{21}=C(21+4-1,4-1)=2024$
    \end{solution}
    \item Suppose no candidate receives a majority of the vote. 
    Repeat part (a).
    \begin{solution}
        Majority vote is if a candidate obtains 13 or more votes. 
        Then, $e_i \le 12$ if no candidate recieves a majority of
        the votes. \\
        $g(x)=(1+x+x^2+\dots+x^{12})^4 = (\dfrac{1-x^{13}}{1-x})^4$\\
        Suppose a candidate gets majority vote, then we have 
        $e'_1+e_2+e_3+e_4=12$ where $e_1 = e'_1 + 13$. So, the
        number of distributions where the candidate obtains majority 
        vote is $C(12+4-1, 4-1)=455$ The coefficient of $x^25$ is 
        $3276 - 4\times 455=1456$.  
    \end{solution}
\end{enumerate}
10. Given one each of $u$ types of candy, two each of $v$ types of 
candy, and three each of $w$ types of candy, find a generating 
function for the number of ways to select $r$ candies.
\begin{solution}
    $g(x) = (1+ x)^u(1+x+x^2)^v(1+x+x^2+x^3)^w$
\end{solution}
16. Find a generating function for the number of integers between 
0 and 999,999 whose sum of digits is r.
\begin{solution}
    $e_1 + e_2 + e_3 + e^4 + e^5 + e^6 = r$, $0\le e_i\le9$\\
    $g(x)=(1+x+x^2+\dots+x^9)^6 = \dfrac{(1-x^{10})^6}{(1-x)^6}$
\end{solution}
Section 6.2\\
2. Find the coefficient of $x^r$ in $(x^5 + x^6 + x^7 +\dots)^8$.
\begin{solution}
    $(x^5 +x^6 + x^7 +\dots) = [(x^5)(1+x+x^2+\dots)]^8
    =x^{40}\dfrac{1}{(1-x)^8}$\\
    $\dfrac{1}{(1-x)^n}= 1+C(1+n-1,1)x+C(2+n-1,2)x^2 +\dots + 
    C(m+n-1,r)x^m + \dots$\\
    $\dfrac{x^{40}}{(1-x)^8}= x^{40}+C(1+8-1,1)x^{41}+
    C(2+8-1,2)x^{42} +\dots + C((r-40)+8-1,r)x^{m+40} + \dots$\\
    The coefficient for $x^r$ for $r\ge 40$ is $C((r-40)+8-1,(r-40))$.
\end{solution}
20. How many ways are there to paint the 10 identical rooms in a 
hotel with five colors if at most three rooms can be painted green, 
at most three painted blue, at most three red, and no constraint 
is laid on the other two colors, black and white?
\begin{solution}
    Max 3 green, 3 blue, 3 red, no constraint on black white. \\
    $e_1 + e_2 + e_3 + e_4 + e_5 = r$, where $e_1, e_2, e_3 \le 3$.\\
    $g(x)= (1+x+x^2+x^3)^3 \cdot (1+x+x^2+\dots)^2 =
    \left(\dfrac{(1-x^4)^3}{(1-x)^5}\right) = \sum g_ix^i$\\
    $(1-x^4)^3 = 1-C(3,1)x^4 + C(3, 2)x^8 + C(3, 3)x^{12}=\sum a_i x^i$\\
    $\dfrac{1}{(1-x)^5} = 
    1+\dots+
    \binom{2+5-1}{2}x^2+\dots+
    \binom{6+5-1}{6}x^6+\cdots + 
    \binom{10+5-1}{10}x^{10}+\cdots=\sum b_i x^i$\\
    $g_{10} = (a_0b_{10} + a_4b_6+a_8b_2) = 
    1\cdot \binom{10+5-1}{10}-
    \binom{3}{1}\cdot\binom{6+5-1}{6}+
    \binom{3}{2}\cdot\binom{2+5-1}{2} = 416$
\end{solution}
22. How many ways are there to get a sum of 25 when 10 distinct 
dice are rolled?
\begin{solution}
    $\sum_{i=1}^{10} e_i = r, 1 \le e_i\le 6$\\
    $g(x) = (x+x^2+x^3+x^4+x^5+x^6)^{10}=x^{10}(1+x+\dots+x^5)^{10} = 
    \dfrac{x^{10}(1-x^6)^{10}}{(1-x)^{10}}$\\
    We need coefficient of $x^{15}$ in $\dfrac{(1-x^6)^{10}}{(1-x)^{10}}$. \\
    $(1-x^6) = 1 - C(10, 1)x^6 + C(10, 2)x^{12} \pm\dots = \sum a_ix^i$\\
    $\dfrac{1}{(1-x)^{10}} = $\\
    $1+\cdots+ 
    C(3+10-1,3)x^3 + \cdots +
    C(9+10-1,9)x^9 + \cdots +
    C(15+10-1, 15)x^{15} +\cdots = \sum b_ix^i$\\
    ${g_15} = (a_0 b_{15} + a_6 b_9 + a_{12}b_3) = 
    1\cdot\binom{15+10-1}{15} +
    \binom{10}{1}\binom{9+10-1}{9} + 
    \binom{10}{2}\binom{3+10-1}{3} = 831204$ ways. 
\end{solution}
\end{document}