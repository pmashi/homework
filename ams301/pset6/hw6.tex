\documentclass[12pt]{article}
\usepackage[utf8]{inputenc}
\usepackage[
    top=        2.5cm, 
    bottom=     2.5cm, 
    left=       2.5cm, 
    right=      2.5cm
]{geometry}
% \usepackage{titling}
\usepackage{amsmath}
\usepackage{amssymb}
\usepackage{enumitem}

\usepackage{titlesec}
\usepackage[many]{tcolorbox}

% Adjust spacing after the chapter title
\titlespacing*{\chapter}{0cm}{-2.0cm}{0.50cm}
\titlespacing*{\section}{0cm}{0.50cm}{0.25cm}

% Indent 
\setlength{\parindent}{0pt}
\setlength{\parskip}{1ex}

% --- Theorems, lemma, corollary, postulate, definition ---
% \numberwithin{equation}{section}

\newtcbtheorem[]{problem}{Solution}%
{enhanced,
	colback=black!5,
	colbacktitle=black!10,
	coltitle=black,
	boxrule=0pt,
	frame hidden,
	borderline west={0.5mm}{0.0mm}{black},
	fonttitle=\bfseries\sffamily,
	breakable,
	before skip=3ex,
	after skip=3ex,
	detach title
}{problem}

\tcbuselibrary{skins, breakable}

% --- You can define your own color box. Just copy the previous \newtcbtheorm definition and use the colors of yout liking and the title you want to use.
% --- solutionbox.tex ---
\usepackage[many]{tcolorbox}

\tcbuselibrary{skins, breakable}

% Define the "solution" environment
\newtcolorbox{solution}{
  enhanced,
  colback=black!3,
  colframe=black!50,
  boxrule=0.5pt,
  arc=2mm,
  left=5mm,
  right=5mm,
  top=2mm,
  bottom=2mm,
  fonttitle=\bfseries\sffamily,
  title=Solution,
  colbacktitle=black!10,
  coltitle=black,
  breakable,
  before skip=2ex,
  after skip=2ex,
  before upper=\strut, 
  detach title
}


\begin{document}
\noindent
AMS 301\\
Ma, Pei \\
Set 6\\

Sec 5.1: 16abc, 30, 36\\
Sec 5.2: 8, 38, 54\\
Sec 5.3: 6, 9, 15, 21, 22\\ 
Sec 5.4: 2, 3ab, 7, 10, 11, 12, 48\\

Section 5.1\\
16. \begin{enumerate}[label=(\alph*)]
    \item How many different outcomes are possible when 
    a pair of dice, one red and one white, 
    are rolled two successive times? 
    \begin{solution}
        Let $R = W = {1, 2, 3, 4, 5, 6}$ where $R$ is the set 
        outcomes of rolling the red die and $W$ is of the white die.\\
        Possible outcomes in one roll can be expressed as $R \times W$, 
        and if we take the cardinality, $O=|R\times W| = |R| \times |W|
        =6*6=36$, which is the number of possble outcomes in one roll.\\
        The number of outcomes in two successive rolls will be 
        $O\times O$, and the number of possible outcomes in two rolls 
        is $|O\times O|=36\times36 = 1296$. 
    \end{solution}
    \item What is the probability that each die shows the 
    same value on the second roll as on the first roll? 
    \begin{solution}
        Since there are 36 possible outcomes for a single roll, 
        there are 36 possible outcomes for two rolls to have the
        same values. So, the probability that each die shows the 
        same value on the second roll as on the first roll is 
        $\dfrac{36}{1296} = 0.46\%$. 
    \end{solution}
    \item What is the probability that the sum of the 
    two dice is the same on both rolls?
    \begin{solution}
        The possible sums are 2 to 12 (11 different outcomes). 
        The possible outcomes for each sum are as following, 
        1 to 6 for sums 2 to 7, and 5 to 1 for sums 8 to 12. 
        So, if the sum of the two rolls are the same, the total
        outcomes for a sum $x$ will be the number of outcomes
        for that sum squared. For example, a sum of 7 has 6 
        outcomes for one roll, so for two rolls to be 
        equal, there are $6^2$ outcomes. If we sum up the outcomes 
        for each sum, $1^2+2^2+...+6^2+5^2+...+1^2=146$. 
        So, the probability that the sum of the two dice is 
        the same on both rolls is $146/1296 = 11.3\%$. 
    \end{solution} 
\end{enumerate}

30. How many times is the digit 5 written when listing all numbers
from 1 to 100,000? Comment on the answers \textbf{(a)} 4, \textbf{(b)} 
$5\times10^4$, and \textbf{(c)} 1 + 10 + 100 + 1000. 
\begin{solution}
    \textbf{(a)} I don't know how they got 4. A singular number 
    55555 has five 5s, which is already greater than 4. \\
    \textbf{(b)} This answer is correct. \\
    \textbf{(c)} If we consider the number of numbers with 5 in 
    the ones place, we can see by observation, that starting from 5, 
    every 10 numbers will have a 5. Since there are 100,000 numbers, 
    this means there are at least 10,000 numbers with at least one 
    digit being 5. 
\end{solution}

36. If two different integers between 1 to 100 inclusive are chosen
at random, what is the probability that the difference of the two 
numbers is 15? 
\begin{solution}
    The minimum integer is 1 in the pair (1, 16) and the maximum is 
    100 in the pair (85, 100). Then, we can see that there are 85 
    pairs. There are $\binom{100}{2}=\frac{100!}{98!2!}=100\cdot99/2
    =4950$ total ways to pick two different integers. So, the 
    probability that two different integers chosen have a difference
    of 15 is $85/4950 = 1.7\%$. 
\end{solution}

Section 5.2 \\
8. There are nine white balls and four red balls in an urn. How
many different ways are there to select a subset of six balls, 
assuming the 13 balls are different? What is the fraction of 
selections with four whites and two reds? 
\begin{solution}
    There are $\binom{13}{6}=\dfrac{13!}{6!(13-6)!}=1716$ ways to 
    select a subset of six balls, assuming the 13 balls are different. \\
    There are $\binom{9}{4}\binom{4}{2}=\dfrac{9!4!}{4!(9-4)!2!(4-2)!}
    =756$ selections with 4 whites and two reds. So, the fraction of 
    selections is $\dfrac{756}{1716}=44.1\%$.  
\end{solution}
38. A student must answer five out of 10 questions on a test, 
including at least two of the first five questions. How many 
subsets of five questions can be answered?
\begin{solution}
    We require two of the first five questions to be chosen, but 
    that does not limit only two of the five to be chosen. So, we must 
    consider the following cases, 2, 3, 4, or 5 of the first five 
    questions get answered. \\
    Case 2: Choose 2 of the first 5, then choose 3 of the last 5. \\
    Case 3: Choose 3 of the first 5, then choose 2 of the last 5. \\ 
    Case 4: Choose 4 of the first 5, then choose 1 of the last 5. \\
    Case 5: Choose 5 of the first 5. \\
    $
    \dbinom{5}{2}\dbinom{5}{3}
    +\dbinom{5}{3}\dbinom{5}{2}
    +\dbinom{5}{4}\dbinom{5}{1}
    +\dbinom{5}{5}\dbinom{5}{0}
    =2\cdot\dfrac{5!5!}{2!2!3!3!}
    +\cdot\dfrac{5!5!}{1!1!4!4!}
    +\cdot\dfrac{5!5!}{0!0!5!5!}
    =200 + 25 + 1 = 226
    $ subsets. 
\end{solution}
54. How many arrangements of MATHEMATICS are there in which each consonant
is adjacent to a vowel?
\begin{solution}
    There are 4 vowels: A, E, A, I; and 7 consonants: M, T, H, M, T, C, S 
    Let c denote consonant and v denote consonant. 
    We can have the following forms: \\
    cvc cvc cvc cv = \{1, 2, 2, 2, 0\} (x2 for symmetry)\\
    cvc cvc cvc vc = \{1, 2, 2, 1, 1\} (x2 for symmetry)\\
    cvc cvc vcc vc = \{1, 2, 1, 2, 1\}\\
    And by symmetry and the rule that the ends in the 
    set must be 1, we have 5 different forms. \\
    The arrangements for the vowels is $\dfrac{4!}{2!}=12$ 
    since we have 2 A's, and the arrangements for the 
    consonants is $\dfrac{7!}{2!2!}=1260$ since we have 2 M's 
    and 2 T's. \\

    So, there are a total of $5\times12\times1260=75600$ 
    arrangements of MATHEMATICS where each consonant is 
    adjacent to a vowel. 
\end{solution}
Section 5.3 \\
6. If four identical dice are rolled, how many different outcomes can 
be recorded?    
\begin{solution}
    There are 6 options and we need a combination of 4 numbers. \\
    So the number of different outcomes is $\dbinom{6+4-1}{4}=126$. 
\end{solution}
9. How many ways are there to pick a selection of coins from \$1 worth 
of identical pennies, \$1 worth of identical nickels, and \$1 worth of 
identical dimes if\\
(a) You select a total of 9 coins?
\begin{solution}
    There are 100 pennies, 20 nickels, and 10 dimes for selection. \\
    $p + n + d = 9; \quad p\le100, n\le20, d\le10$. \\
    $\dbinom{9+3-1}{9} = \dfrac{11!}{9!2!} = 55$ selections.
\end{solution}
(b) You select a total of 16 coins?
\begin{solution}
    There are 100 pennies, 20 nickels, and 10 dimes for selection. \\
    Assume, we have an unlimited amount of each coin. Then, there are 
    $\dbinom{16+3-1}{16}=153$ different selections.\\
    Now, if we consider the restriction of 10 dimes, we have to remove 
    the additional selections where there are more than 10 dimes, hence
    $p + n + (d-11) = 5$. $\binom{5+3-1}{5}=21$. So, there are 21 
    selections where there are more than 10 dimes, which means there 
    are 153-21=132 selections of 16 coins. \\

    Comment on the answers: 
    \begin{enumerate}[label=(\alph*)]
        \item $\dbinom{16+3-1}{16}$\\
        this solution does not consider the restriction. 
        \item $\displaystyle \sum_{k=0}^{10}\dbinom{16}{k}
        \dbinom{(16-k)+2-1}{(16-k)}$\\   
        This answer does not make any sense. The sum appears to consider
        the possibilities if there were k dimes. 
        \item $\dbinom{16+3-1}{16}-\displaystyle\sum_{k=11}^{16}
        \dbinom{16}{k}\dbinom{(16-k)+2-1}{(16-k)}$\\
        $<\dbinom{18}{16} - \dbinom{16}{11}\dbinom{6}{5}=-26055$.
        This solution is negative, which doesn't make sense. There cannot 
        be a negative amount of selections. 
    \end{enumerate}
\end{solution}
15. How many 8-digit sequences are there involving exactly six 
different digits?
\begin{solution}
    There are 10 different digits. You can have a triplet of a digit, or 
    a two pairs of digits for repetitions. \\
    For a triplet digit, 
    there are $\dbinom{10}{6} = 210$ sets of digits, 
    6 choices for the repeating digit,  
    and $\dfrac{8!}{3!}=6720$ different sequences.\\
    For the two pairs, 
    there are $\dbinom{10}{6} = 210$ sets of digits, 
    $\dbinom{6}{2}=15$ repeating pairs,  
    and $\dfrac{8!}{2!2!}=10080$ sequences. \\
    So there are a total of $210\times6\times6720 + 210\times15\times10080
    = 42487200$
    different sequences. 
\end{solution}
21. How many arrangements of the letters in MATHEMATICS are there in which
TH appear together but the TH is not immediately followed by an E (not THE)?
\begin{solution}
    Consider "THE" as one character. Note that there are 2 copies of M, 
    two copies of A, and two copies of T. 
    So there are 2 ways to arrange "THE". There are 11 letters, but if 
    "THE" is considered one letter, then there are $11-3+1=9$ letters. 
    So, there must be $\dfrac{9!}{2!2!} = 90720$ arrangements where "THE"
    are put 
    together (We divide by $2!2!$ for the two copies of M and A). \\
    Consider "TH" as one character, and the copy of T. Following the same
    logic, there are 10 letter, so there must be $10!/2 = 907200$ 
    arrangements. \\
    So, there are $907200-90720=816480$ arrangements of the letters in MATHEMATICS 
    which TH appear together, but not followed by an E. 
\end{solution}
22. How many arrangements of the letters in PEPPERMILL are there with \\
(a) The M appearing to the left of all the vowels?
\begin{solution}
    There are 3 vowels. 
\end{solution}
(b) The first P appearing before the first L?
\begin{solution}
    There are 3 P's, 2 L's, and 10 total letters. \\
    
\end{solution}
Section 5.4 \\
2. How many ways are there to distribute 18 different toys among four children?
(a) Without restrictions?
(b) If two children get seven toys and two children get two toys?
\begin{solution}
    
\end{solution}
3. In a bridge deal, what is the probability that:
(a) West has four spades, two hearts, four diamonds, and three clubs?
(b) North and South have four spades, West has three spades, and East has two
spades?
\begin{solution}
    
\end{solution}
7. How many ways are there to arrange the letters in VISITING with no pair of
consecutive Is?
\begin{solution}
    
\end{solution}
10. How many ways are there to arrange the 26 letters of the alphabet so that no pair
of vowels appear consecutively (Y is considered a consonant)?
\begin{solution}
    
\end{solution}
11. If you flip a coin 18 times and get 14 heads and four tails, what is the probability
that there is no pair of consecutive tails?
\begin{solution}
    We can rephrase this question to an arrangement of letters question, 
    where H is heads and T is tails. Then, the arrangement 
\end{solution}
12. How many integer solutions are there to $x_1 + x_2 + x_3 + x_4 + x_5 
= 31$ with 
\begin{enumerate}[label=(\alph*)]
    \item $x_i \ge 0$ 
    \item $x_i > 0$ 
    \item $x_i \ge i(i = 1, 2, 3, 4, 5)$
\end{enumerate}
\begin{solution}
    
\end{solution}
48. How many arrangements of the letters in INSTRUCTOR have all of the
following properties simultaneously?
\begin{enumerate}[label=(\alph*)]
    \item The vowels appearing in alphabetical order
    \item At least 2 consonants between each vowel
    \item Begin or end with the 2 Ts (the Ts are consecutive)
\end{enumerate}
\begin{solution}
    There are a total of 10 letters, 3 are vowels (I, U, O) and 7 are 
    consonants. \\
    To satisfy the first condition, the order of the vowels must be: \\
    $r_1,I, r_2, O, r_3, U, r_4$, where $r_i$ is a region of consonants. \\
    The second condition requires that region $r_2$ and $r_3$ to consist 
    of two characters. $r_1, I, c_1, c_2, O, c_3, c_4, U, r_4$. \\
    The third condition requires either $r_1$ or $r_4$ to contain TT. So 
    the forms are: \\
    $T, T, I, c_1, c_2, O, c_3, c_4, U, r_4$ \\
    $r_1, I, c_1, c_2, O, c_3, c_4, U, T, T$ \\
    This leaves the final regions to be 1 character long, or 0, with 
    another consonant adjacant to TT, either $cTT$ or $TTc$. \\
    So, there must be $(\text{TT are at }r_1 \text{ or } r_4) \times 
    (\text{c is(n't) with the TT }) = 2 \times 2 = 4$ 
    forms. \\
    Now, if we arrange the remaining consonants, there will be $5!=120$ 
    permutations. Hence, there will be $120 \times 4 = 480$ arrangements 
    of the letters in INSTRUCTOR with the properties. 
\end{solution}

\end{document}