\documentclass[12pt]{article}
\usepackage[utf8]{inputenc}
\usepackage[
    top=        2.5cm, 
    bottom=     2.5cm, 
    left=       2.5cm, 
    right=      2.5cm
]{geometry}
\usepackage{titling}
\usepackage{amsmath}
\usepackage{amssymb}
\usepackage{enumitem}

\usepackage{titlesec}
\usepackage[many]{tcolorbox}

% Adjust spacing after the chapter title
\titlespacing*{\chapter}{0cm}{-2.0cm}{0.50cm}
\titlespacing*{\section}{0cm}{0.50cm}{0.25cm}

% Indent 
\setlength{\parindent}{0pt}
\setlength{\parskip}{1ex}

% --- Theorems, lemma, corollary, postulate, definition ---
% \numberwithin{equation}{section}

\newtcbtheorem[]{problem}{Solution}%
{enhanced,
	colback=black!5,
	colbacktitle=black!10,
	coltitle=black,
	boxrule=0pt,
	frame hidden,
	borderline west={0.5mm}{0.0mm}{black},
	fonttitle=\bfseries\sffamily,
	breakable,
	before skip=3ex,
	after skip=3ex,
	detach title
}{problem}

\tcbuselibrary{skins, breakable}

% --- You can define your own color box. Just copy the previous \newtcbtheorm definition and use the colors of yout liking and the title you want to use.
% --- solutionbox.tex ---
\usepackage[many]{tcolorbox}

\tcbuselibrary{skins, breakable}

% Define the "solution" environment
\newtcolorbox{solution}{
  enhanced,
  colback=black!3,
  colframe=black!50,
  boxrule=0.5pt,
  arc=2mm,
  left=5mm,
  right=5mm,
  top=2mm,
  bottom=2mm,
  fonttitle=\bfseries\sffamily,
  title=Solution,
  colbacktitle=black!10,
  coltitle=black,
  breakable,
  before skip=2ex,
  after skip=2ex,
  before upper=\strut, 
  detach title
}


\begin{document}
\noindent
AMS 301\\
Ma, Pei \\
Set 6\\

Sec 5.1: 16abc, 30, 36\\
Sec 5.2: 8, 38, 54\\
Sec 5.3: 6, 9, 15, 21, 22\\ 
Sec 5.4: 2, 3ab, 7, 10, 11, 12, 48\\

Section 5.1\\
16. \begin{enumerate}[label=(\alph*)]
    \item How many different outcomes are possible when 
    a pair of dice, one red and one white, 
    are rolled two successive times? 
    \begin{solution}
        Let $R = W = {1, 2, 3, 4, 5, 6}$ where $R$ is the set 
        outcomes of rolling the red die and $W$ is of the white die.\\
        Possible outcomes in one roll can be expressed as $R \times W$, 
        and if we take the cardinality, $O=|R\times W| = |R| \times |W|
        =6*6=36$, which is the number of possble outcomes in one roll.\\
        The number of outcomes in two successive rolls will be 
        $O\times O$, and the number of possible outcomes in two rolls 
        is $|O\times O|=36\times36 = 1296$. 
    \end{solution}
    \item What is the probability that each die shows the 
    same value on the second roll as on the first roll? 
    \begin{solution}
        Since there are 36 possible outcomes for a single roll, 
        there are 36 possible outcomes for two rolls to have the
        same values. So, the probability that each die shows the 
        same value on the second roll as on the first roll is 
        $\dfrac{36}{1296} = 0.46\%$. 
    \end{solution}
    \item What is the probability that the sum of the 
    two dice is the same on both rolls?
    \begin{solution}
        The possible sums are 2 to 12 (11 different outcomes). 
        The possible outcomes for each sum are as following, 
        1 to 6 for sums 2 to 7, and 5 to 1 for sums 8 to 12. 
        So, if the sum of the two rolls are the same, the total
        outcomes for a sum $x$ will be the number of outcomes
        for that sum squared. For example, a sum of 7 has 6 
        outcomes for one roll, so for two rolls to be 
        equal, there are $6^2$ outcomes. If we sum up the outcomes 
        for each sum, $1^2+2^2+...+6^2+5^2+...+1^2=146$. 
        So, the probability that the sum of the two dice is 
        the same on both rolls is $146/1296 = 11.3\%$. 
    \end{solution} 
\end{enumerate}

30. How many times is the digit 5 written when listing all numbers
from 1 to 100,000? Comment on the answers \textbf{(a)} 4, \textbf{(b)} 
$5\times10^4$, and \textbf{(c)} 1 + 10 + 100 + 1000. 
\begin{solution}
    \textbf{(a)} I don't know how they got 4. A singular number 
    55555 has five 5s, which is already greater than 4. \\
    \textbf{(b)} This answer is correct. \\
    \textbf{(c)} If we consider the number of numbers with 5 in 
    the ones place, we can see by observation, that starting from 5, 
    every 10 numbers will have a 5. Since there are 100,000 numbers, 
    this means there are at least 10,000 numbers with at least one 
    digit being 5. 
\end{solution}

36. If two different integers between 1 to 100 inclusive are chosen
at random, what is the probability that the difference of the two 
numbers is 15? 
\begin{solution}
    The minimum integer is 1 in the pair (1, 16) and the maximum is 
    100 in the pair (85, 100). Then, we can see that there are 85 
    pairs. There are $\binom{100}{2}=\frac{100!}{98!2!}=100\cdot99/2
    =4950$ total ways to pick two different integers. So, the 
    probability that two different integers chosen have a difference
    of 15 is $85/4950 = 1.7\%$. 
\end{solution}

Section 5.2 \\
8. There are nine white balls and four red balls in an urn. How
many different ways are there to select a subset of six balls, 
assuming the 13 balls are different? What is the fraction of 
selections with four whites and two reds? 
\begin{solution}
    There are $\binom{13}{6}=\dfrac{13!}{6!(13-6)!}=1716$ ways to 
    select a subset of six balls, assuming the 13 balls are different. \\
    There are $\binom{9}{4}\binom{4}{2}=\dfrac{9!4!}{4!(9-4)!2!(4-2)!}
    =756$ selections with 4 whites and two reds. So, the fraction of 
    selections is $\dfrac{756}{1716}=44.1\%$.  
\end{solution}
38. A student must answer five out of 10 questions on a test, 
including at least two of the first five questions. How many 
subsets of five questions can be answered?
\begin{solution}
    We require two of the first five questions to be chosen, but 
    that does not limit only two of the five to be chosen. So, we must 
    consider the following cases, 2, 3, 4, or 5 of the first five 
    questions get answered. \\
    Case 2: Choose 2 of the first 5, then choose 3 of the last 5. \\
    Case 3: Choose 3 of the first 5, then choose 2 of the last 5. \\ 
    Case 4: Choose 4 of the first 5, then choose 1 of the last 5. \\
    Case 5: Choose 5 of the first 5. \\
    $
    \binom{5}{2}\binom{5}{3}
    +\binom{5}{3}\binom{5}{2}
    +\binom{5}{4}\binom{5}{1}
    +\binom{5}{5}\binom{5}{0}
    =2\cdot\dfrac{5!5!}{2!2!3!3!}
    +\cdot\dfrac{5!5!}{1!1!4!4!}
    +\cdot\dfrac{5!5!}{0!0!5!5!}
    =200 + 25 + 1 = 226
    $ subsets. 
\end{solution}
54. How many arrangements of MATHEMATICS are there in which each consonant
is adjacent to a vowel?
\begin{solution}
    There are 4 vowels: A, E, A, I; and 7 consonants: M, T, H, M, T, C, S 
    Let c denote consonant and v denote consonant. 
    We can have the following forms: \\
    cvc cvc cvc cv = \{1, 2, 2, 2, 0\} (x2 for symmetry)\\
    cvc cvc cvc vc = \{1, 2, 2, 1, 1\} (x2 for symmetry)\\
    cvc cvc vcc vc = \{1, 2, 1, 2, 1\}\\
    And by symmetry and the rule that the ends in the 
    set must be 1, we have 5 different forms. \\
    The arrangements for the vowels is $\dfrac{4!}{2!}=12$ 
    since we have 2 A's, and the arrangements for the 
    consonants is $dfrac{7!}{2!2!}=1260$ since we have 2 M's 
    and 2 T's. \\

    So, there are a total of $5\times12\times1260=75600$ 
    arrangements of MATHEMATICS where each consonant is 
    adjacent to a vowel. 
\end{solution}
Section 5.3 \\
6. If four identical dice are rolled, how many different outcomes can 
be recorded?
\begin{solution}
    4 identical - xxxx : $\binom{6}{1} = 6$ \\
    3 identical 1 random - xxxy : $\binom{6}{2} = 15$\\
    2 identical 2 different - xxyz : $\binom{6}{3} = 20$\\
    2 pairs of 2 identical - xxyy : $\binom{6}{2} = 15$\\
    all different - wxzy : $\binom{6}{4} = 15$\\
    
\end{solution}
9. How many ways are there to pick a selection of coins from \$1 worth 
of identical pennies, \$1 worth of identical nickels, and \$1 worth of 
identical dimes if
(a) You select a total of 9 coins?
(b) You select a total of 16 coins?
\begin{solution}
    There are 100 pennies, 20 nickels, and 10 dimes for selection. 
\end{solution}
15. How many 8-digit sequences are there involving exactly six 
different digits?
\begin{solution}
    
\end{solution}
21. How many arrangements of the letters in MATHEMATICS are there in which
TH appear together but the TH is not immediately followed by an E (not THE)?
\begin{solution}
    
\end{solution}
22. How many arrangements of the letters in PEPPERMILL are there with
(a) The M appearing to the left of all the vowels?
(b) The first P appearing before the first L?
\end{document}